\chapter{Android Build System}

Ein Build setzt sich aus folgenden Schritten zusammen:
\begin{enumerate}
	\item Check out code
	\item Resolve dependencies
	\item Compile
	\item Static code analysis
	\item Run tests
	\item Check coverage
	\item Obfuscate
	\item Generate documentation
	\item Create Artifacts
	\item Deploy
\end{enumerate}
In Java sind folgende Buildsysteme verbreitet:
\begin{tabbing}
	\hspace{4.2cm}\=\hspace{7cm}\=\kill
	\textbf{Ant} \> \textbf{Maven} \> \textbf{Gradle} \\ 
	Deklarativ (XML) \> Deklarativ (XML ) \> Deklarativ (DSL) \\ 
	Ant Tasks \>  \> Scriptable Tasks + Lifecycle \\ 
	Kein Projektmodell \> Projektmodell (POM) \> Projektmodell \\ 
	Keine Konvention \> Konventionen(Projektstruktur, Lifecycle)  \> Konventionen(Projektstruktur) \\ 
	Kein Lifecycle \> Plugins \> Plugins \\ 
	Kein Dependency-Mgmt \> Dependency-Mgmt \> Dependency-Mgmt \\ 
	Kein Skripting \> Kein Skripting \> Skripting
\end{tabbing}
Gradle Buildskripts enthalten Task-Definitionen, Konfigurationen von Modellobjekten und Tasks sowie ausführbaren Code. Ein Gradle-Build läuft in 2 Phasen ab:
\begin{enumerate}
	\item Konfiguration des Builds (Erstellung Abhängigkeitsgraph)
	\item Ausführung des Builds (d.h. von spezifischen Tasks)
\end{enumerate}
Ein Gradle-Skript kann Plugins anwenden. Plugin wird als Abhängigkeit hinzugefügt (im \texttt{buildscript}-Block)und mit \texttt{apply} aktiviert. Plugins definieren neue Modellobjekte, fügen einem Projekt weitere Tasks hinzu und konfigurieren Buildparameter gemäss Konventionen. Das Android Plugin definiert u.a. die folgenden Tasks:
\begin{itemize}
	\item \texttt{assemble}
	\item \texttt{check} $\rightarrow$ \texttt{lint}
	\item \texttt{test}
	\item \texttt{build} $\rightarrow$ \texttt{assemble}, \texttt{check}
	\item \texttt{install}
	\item \texttt{clean}
\end{itemize}
Zusätzlich wird noch das Modellobjekt \texttt{android} definiert.

Android-Applikationen können grundsätzlich aus mehreren Modulen bestehen. Im einfachsten Fall gibt es ein einzelnes \emph{app} Modul mit einem eigenen Build-File. Auf Projekt-Ebene gibt es noch ein \emph{root} Build-File, welches die gemeinsame Build-Konfigurationen definiert. Der Android Build kann von derselben Applikation mehrere Varianten erstellen. Es gibt Buils Types und Product Flavors. Build Types werden typischerweise in \texttt{debug} und \texttt{release} unterschieden (ProGuard, debug-Flag usw.). Product Flavors erlauben mehrere Varianten derselben App zu erstellen (z.B. Funktionalitätsunterschied Free/Paid).

Gradle bietet zudem noch einen Daemon an. Dieser cached das Modell in-memory und so muss nicht immer wieder geparst werden, was zu einem schnelleren Build führt.